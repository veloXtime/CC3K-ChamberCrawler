\documentclass[11pt]{article}

% page setting
\usepackage[letterpaper, portrait, margin=0.7in]{geometry} \usepackage{parskip}

% 
\usepackage{geometry} 
\geometry{ 
	letterpaper, 
	total={170mm,257mm}, 
	left=27mm,
	right=27mm, 
	top=20mm, 
	bottom=30mm }

% coloring
\usepackage[dvipsnames]{xcolor}

% font setting
%\usepackage[10pt]{moresize}
%\usepackage[utf8]{inputenc}

% include graphics
\usepackage{graphicx} \graphicspath{ {\string~/Documents/deposit/} }

% latex math packages
\usepackage{latexsym,amsfonts,amssymb,amsthm,amsmath}

% graphing
\usepackage{graphpap}
%\usepackage[dvips]{graphics}
\usepackage{tkz-euclide}

% miscellaneous
\usepackage{enumitem} \usepackage{cases}

% clickable content
\usepackage{hyperref} \hypersetup{ colorlinks=true, linktoc=all,
linkcolor=black,}

\usetkzobj{all}

% for taking notes
\theoremstyle{plain} \newtheorem{theorem*}{Theorem}[subsection]
\newtheorem{theorem}{Theorem}[subsection]
\newtheorem{thm}{\textit{Theorem}}[subsection]
\newtheorem{definition}{Definition}[subsection]
\newtheorem{lemma}{Lemma}[subsection]
\newtheorem{conjecture}{Conjecture}[subsection]
\newtheorem{proposition}{Proposition}[subsection]
\newtheorem{example}{Example}[subsection]
\newtheorem{corollary}{Corollary}[subsection]
\newtheorem{algorithm}{Algorithm}[subsection]

% Blackboard bold letters \newcommand {⟨cmd⟩} [⟨num⟩] [⟨default⟩]
% {⟨definition⟩} 

\newcommand{\set}[2]{\{#1\,:\,\text{#2}\}} \newcommand{\tup}[1]{\mathbf{#1}}
\newcommand{\sfP}{\mathsf{P}} \newcommand{\M}{\mathsf{M}}
\newcommand{\dfeq}{\stackrel{\mathrm{def}}{=}} \newcommand{\ra}{\rightarrow}
\newcommand{\la}{\leftarrow} \newcommand{\lra}{\leftrightarrows}
\newcommand{\Span}{\mathrm{span}} \newcommand{\scrP}{\mathscr{P}}
\newcommand{\rank}{\mathrm{rank}} \newcommand{\nullity}{\mathrm{nullity}}
\newcommand{\Col}{\mathrm{Col}} \newcommand{\Row}{\mathrm{Row}}
\newcommand{\tr}{\mathrm{tr}} \let\emptyset\varnothing

\newcommand{\abs}[1]{\left| #1 \right|}

\newcommand{\mC}{{\mathbb{C}}} \newcommand{\mN}{{\mathbb{N}}}
\newcommand{\mQ}{{\mathbb{Q}}} \newcommand{\mR}{{\mathbb{R}}}
\newcommand{\mZ}{{\mathbb{Z}}} \newcommand{\mF}{{\mathbb{F}}}

\newcommand{\Mod}[1]{\ (\mathrm{mod}\ #1)}

\newcommand{\de}{\delta} \newcommand{\ep}{\varepsilon}
\renewcommand{\phi}{\varphi}

\newcommand{\dlim}{\displaystyle\lim\limits}
\newcommand{\dsum}{\displaystyle\sum\limits}

\DeclareMathOperator{\Dim}{dim} \DeclareMathOperator{\Dom}{dom}
\DeclareMathOperator{\Img}{img}

\newcommand{\elemop}[1]{\stackrel{#1}{\longrightarrow}}
\newcommand{\elemoprule}[2]{\underset{(#1)}{\stackrel{#2}{\longrightarrow}}}
\newcommand{\cof}[3]{\widetilde{#1}_{#2 #3}} \newcommand{\sign}{\mathrm{sign}}

\begin{document}

\begin{center}
{\LARGE \textbf{CS 246 Final Project Design}}\\
\vspace{0.15 in}
Xinyan Lin, Hanyu Xu, Rivers Chen\\
\vspace{0.07 in}
August, 2020
\end{center}

\vspace{0.3 in}
\section{OVERVIEW}
The game of ChamberCrawler3000 is a roguelike game in which the player’s goal
is to reach the exit at the top floor of the five-floor map.  

The game follows the OOP design principle, where all actions from clients
should be called in the GameController class.  


\section{DESIGN}
\subsection{System Components}
%\begin{description}
%\item 

\subsubsection{Game Element}

All objects are inherited under an abstract class \texttt{GameElement}, which
has game element's x and y positions and their char to display as attributes.
\texttt{GameElement} has two direct subclasses \texttt{Living} and
\texttt{NonLiving}.


\subsubsection{Living Characters}

The \texttt{Living} class inherits directly from \texttt{GameElement} subclass,
and it an abstract superclass for \texttt{PlayerCharacter} and
\texttt{EnemyCharacter}. \texttt{Living} contains 
character's race, HP, Def, Atk
as attributes, and provides their corresponding accessors and mutators.


The \texttt{PlayerCharacter} class and \texttt{EnemyCharacter} class are 
implemented using inheritance. 
Since they are both classes for living elements with common
features like hp, atk, def, and the ability to attack others, the common
behaviours of \texttt{PlayerCharacter} and \texttt{EnemyCharacter}
are put in the base class Living. 
Then, both PlayerCharacter and EnemyCharacter inherit from the Living
class, but also have extra features different from the base class. Under
PlayerCharacter and EnemyCharacter, each distinct race is implemented as a
subclass of them. 


To implement different races of player characters, the different basic stats
(hp, atk, def) of different races are initialized in the constructor of each
race respectively. For the special ability of different races, the template
method design pattern is applied. For example, shade has its score magnified by
1.5. This is achieved by overriding the score accessor of
\texttt{PlayerCharacter} in the \texttt{Shade} class. 
Instead of returning the actual score, the returned score
value will be multiplied by 1.5. For another example, goblin gains 5 gold for
every slained enemy. This is achieved by overriding the slain function of
\texttt{PlayerCharacter} in the \texttt{Goblin}class. 
Beside the normal function flow, 5 extra
gold is added to the score of the goblin.
Using the template method allows each race to override and modify certain steps
to implement their special abilities,
but also reuses code for common features. The implementation of different
enemy character races is similar to player characters, 
which also applies template method.

\subsubsection{NonLiving Items}

The \texttt{Potion} classes use the decorator design pattern. 
The potions player
character drinks exist as decorators to the player character’s status, and PC
holds a pointer to a concrete component class of “no effect” potion initially.
This way potions can be easily added and removed when needed, and their effects
can be calculated independent of player character. Since potions affecting
attack and defense are only effective in the current floor, they are removed
when the PC reaches the next floor.  

The \texttt{Treasure}
class uses a similar design as the potion class. For each of the
gold size, a specific enum class type exists in the header file which integer
value corresponds to its size. When constructing a treasure, the client does
not need to know necessarily the actual size of the treasure type. Unlike the
\texttt{Potion}
class, the player character is responsible for adding the value of the
treasure to its score field. Since the Dragon Hoard can only be picked up after
the dragon guarding it is slain, a dragon hoard will return 0 from its
getAmount method, indicating that it cannot be picked up yet.  

\subsubsection{Architecture}

Architecture class, named \texttt{Architect}, like \texttt{Living} and
\texttt{NonLiving} is also a direct subclass of \texttt{GameElement}. 
Except the three attributes inherited from \texttt{GameElement}, 
\texttt{Architect} also contains \textsf{chamberInd} as an attribute, 
the use of \textsf{chamberInd} will be further explained in the GameController
and Board setup section.
When initializing Architectures, the constructor takes four parameters.
The x, y coordinate, their char for display, and also \textsf{chamberInd}.
The \textsf{chamberInd} for each floor tiles (`\ .\,') is the index of chamber
it is in, and other architectures it's just the default value $0$.

\subsubsection{Board}

All gameElements are stored in \texttt{Board} as shared pointers. 

\texttt{Board} contains a 3d vector \textsf{floor}, which each
\textsf{floor[i][j]} represents a square on the board, and is implemented 
as an 1-dimensional vector of game element shared pointers.
The first position is always a \texttt{Architect}, and any other game elements
``standing" on this position is pushed back onto this 1-dimensional vector.

\texttt{Board} also contains vector of shared pointer of \texttt{EnemyCharacter}
named \textsf{enemyList}, {\color{red}
used when enmey moves(discussed in section ... ).}



\subsubsection{GameController}

\texttt{GameController} acts as a controller of the whole game, it is a 
friend class of \texttt{Board}. The functions in \texttt{GameController}
will access the game elements on \textsf{floor} to handle tasks such as
setup of \textsf{floor}, spawn of PC and enemies, move PC and enemies, etc.
The specific design of how these are complished will be in
{\color{red} in section}.

\texttt{GameController} holds a shared pointer to our player character
and also the x and y position of the stair of current floor.

\subsubsection{Display}





\subsection{Features}

\subsubsection{Setup}

If we are at the initial level, 
the game will ask player to give a valid player character race char.

Each level, \texttt{floor} in the board will be reset, so all game elements of 
last floor will be discarded except pc. The \textsf{enmeyList} will be reset.

If no command line argument is provided, then, a default layout of floor which
only contains architectures will be used to build to fstream. 
\texttt{gameController} will use this fstream to setup an initial empty
board, then use the char player provide to spawn PC. Afterwards, 
\texttt{gameController} will spawn other game elements, when spawning 
enemy, each enemy will be added to the \textsf{enemyList}. And, each
dragon hoard will have its dragon as a field.

If a single command line argument which is the name of a file that specifies
the layout of each of the 5 floors is provided, then, \textsf{readFloor}
of \texttt{gameController} will be called to change each char in the file to
an object on our board. 


\subsubsection{ During the game }

The player may choose to move pc, attack enemy, 


\subsubsection{Move}

\textit{Combat: }

When implementing the combat system between player character and enemy
character, observer, visitor and template method are all in use. When the
player character gets within 1 block radius to an enemy, observer pattern is
used to notify the enemy so that the enemy will attack the player character.
Since the damage dealt by an enemy varies based both the race of the enemy and
the player character, visitor pattern and double dispatch are used to implement
an attack. Whenever a character A is trying to attack a character B, it calls
the function A.attack(B), then A.attack(B) calls B.attackedBy(A), which is an
equivalent visit function that behaves differently based on the concrete class
type of A (races of A in this case). Such design pattern makes the combat
system clear and straight forward. For different races, damage will be
calculated differently. Furthermore, some races have special interaction
between each other. For example, orcs do 50\% more damage to goblins. This is
where template method is applied. In the Goblin class, its function
attackedBy(Orc) is overwritten such that the normal damage is multiplied by
1.5. 

PC-unrelated Control: 

board 

User Input Interpretation: 

Main.cc? 

Game Flow Control: 

display 
%\end{description}


\section{Resilience to Change}

The living class involves player character and enemy character. It is designed
to follow high cohesion and low coupling. In the living class, each distinct
race has its own header and implementation files. Thus, a header and an
implementation serves on a single purpose of implementing the specific race. It
demonstrates the high cohesion in our code. The communication between player
character and enemy character happens in combat only. In a combat, function
calls only takes others in as basic parameters without knowing the actual
implementation. It demonstrates the low coupling in our code as different
modules has very low dependencies between each other. 

Since the potion subclasses use the decorator design pattern, which allows us
to easily add new types of potions to the game, as well as designing more
complicated types of potions, such as ones that have dynamic effects depending
on the progress of the game. Its functionality is also independent from the
player character, which is easier to make changes to the potion classes only. 

The treasure class uses a enum type to define the various size of gold, if
other sizes of gold are need, or if the current size of gold has to be changed,
one simply has to define additional members or modify existing ones in the enum
type and pass the type into the constructor of treasure. The existing code of
generating treasures may not need change at all. If other types of treasure are
need, subclasses of Treasure can be added to accommodate the change. How
treasures are picked up by Player character is defined on the PC’s side, who is
a friend of treasure class, thus no other changes are needed. 

The display class handles all the standard message the game has by having a set
of flags corresponding to various possible states of the game. It also
dynamically read in the game board when printing its content, and thus there is
no need to change the code of display if other player characters and/or enemies
are added to the board.  


Answers to Questions 

How could your design your system so that each race could be easily generated?
Additionally, how difficult does such a solution make adding additional races? 

The answer for this question is basically the same as answered in due date 1
design. 

To generate each race easily, we would apply the template method when creating
races. We would create a class named PlayerCharacter as the abstract base
class. A player character must be one of the race type, thus an object of the
parent PC (without race) will never be initialized, which means PC will have
pure virtual method (discussed later). Then, based on the template method, we
will make each race a distinct sub class of PC.  

For a player character of any race, it has common attributes of HP, ATK, DEF
and MAX HP. These are common behaviors of all races. Thus, these attributes
will be put into the parent class PC. Also, there will be accessor and mutator
methods for each field in the parent class. PC will also have methods called
“attack” and “attackedBy”. “Attack” is a method called upon the PC attacking an
enemy whereas “attackedBy” is a method called upon the PC being attacked by an
enemy. 

Such solution makes adding additional races easily. If an additional race is to
be added, then we would make the race a new sub class inherited from the base
class PC. We would create a header file and a separate implementation file for
the race. We can override any method required in the sub class. Doing so
results in high cohesion and low coupling. The new header and implementation
file serves solely for the purpose of creating a new race. Also, such
implementation will not affect the base class or any other races. 

How does your system handle generating different enemies? Is it different from
how you generate the player character? Why or why not? 

For how our system handle generating different enemies, the answer is same as
answered in due date 1 design. 

To generate different enemies, we would also use template method. We would
create a class named Enemy as the abstract base class. An enemy must be one of
the race type, thus an object of the parent Enemy (without race) will never be
initialized, which means Enemy will have pure virtual method (destructor
possibly). Then, based on the template method, we will make each race a
distinct sub class of Enemy.  

For an enemy of any race, it has common attributes of HP, ATK, DEF. These are
common behaviors of all races. Thus, these attributes will be put into the
parent class Enemy. Also, there will be accessor and mutator methods for each
field in the parent class. Enemy will also contain methods called “attack” and
“attackedBy”. “Attack” is a method called upon the enemy attacking a PC whereas
“attackedBy” is a method called upon the enemy being attacked by a PC. 

For the difference between generating player character and enemy character,
it’s different from our due date 1 design. 


We have a class called GameController which is responsible for game flow and
management of pc-related behaviours. GameController is the class responsible
for generating player character. When user input their desired race,
GameController has a function spawnPC() that takes in the race and generates
the required race PC.  

We have a class called Board which is responsible for game component generation
and management of pc-unrelated behaviours.  At the beginning of initialization
of every floor, enemies will be spawned. For the enemy generation, the Board
will call its method “spawnEnemy”. SpawnEnemy has a helper named spawnOneEnemy
that can spawn one specified enemy. SpawnEnemy determines an enemy race based o
n the probability distribution, and calls spawnOneEnemy to generate such enemy.
It repeats the process 20 times to get 20 random enemies. 

On due date 1, we proposed to use Board class to controll the entire game flow.
But in actual implementation, we decided to have Board and GameController
taking different responsibilities of the game flow. Doing so increases the
cohesion within the classes. We put PC generation in GameController, as its
generation is related to user inputs. We put enemy generation in Board, as it
belongs to part of the floor initialization and is unrelated to user’s
behaviours. Such reason causes the generation of player character and enemy
character to be different. 

How could you implement the various abilities for the enemy characters? Do you
use the same techniques as for the player character races? Explain. 

The answer for this question is slightly different from the answer in due date
1 design. 

The various abilities for the enemy characters will be implemented differently.
We sort the abilities of enemy characters into three categories: attack
priority (for halfling), extra damage (for dwarf, elf, orcs), and others (for
human, dragon).  

First, we would expose our implementation of combat. In a combat, a character A
attacking a character B has the following procedure: calling A.attack(B),
A.attack(B) will call B.attackedBy(A), “attackedBy” calculates the actual
damage and change the HP of B consequently. Here we exploits the observer
pattern and visitor pattern. The observer pattern is used in such way: when a
PC gets within 1 block radius of an enemy, it notifies the enemy so that the
enemy get notified. As long as an enemy gets notified, it tries to attack PC.
The visitor pattern is used so that the damage will be calculated differently
based on the races of player character and enemy character. 

To implement attack priority for halfling, when PC tries to attack a halfling,
a 50% probability of successful attack is implemented in PC’s attack function.
Also, halfling has 100% chance of successful attack, thus the getNotified
function of EnemyCharacter is overridden by halfling. Normally, when an enemy
gets notified by an PC, it has 50% chance of successful attack. For halfling,
the probability check is removed so that an halfling always attacks the PC when
getting notified. 

To implement extra damage, the template method will be applied here. In
default, the base PlayerCharacter class has 7 “attackedBy” methods that each
takes in a different enemy race with respect to the visitor pattern. Howevery,
the enemy races belong to the extra damage categories do extra damage to
specfic PC races. Specifically, vampires react to dwarves, all races except
drow react to elfs, and goblins react to orcs. Thus, we will override the
distinct “attackedBy” for the mentioned enemy race in the character race class.
For example,  when attackedBy takes in an orc, goblin class will override
attackedBy(orc) and special damage calculation will be done. When attackedBy
takes in other races, it follows the default attackedBy defined in the
PlayerCharacter class. 

To implement the ability of human, upon the death of a human, extra gold pile
generation is required. As mentioned before, we have a Board class responsible
for management of pc-unrelated behaviours. Specifically, it is responsible for
removing enemies from the board when the enemies are slained. Thus in Board’s
function enemyDeath(), it checks the race of the dead enemy. If it’s determined
to be a human, two piles of gold will be generated. 

To implement the ability of dragon, its associared dragon hoard maintains a
pointer to its dragon’s address. It means the dragon generation is isolated
from normal enemy generation. Instead, whenever a dragon hoard is generated, a
dragon will be spawned consequently.  

The implementation of enemy abilities is different from of PC races. PC races
has no special interaction with enemies. Thus, they can be implemented through
changing the information of PC (I.e. fields and methods). But most enemy
abilities have special interaction with PCs (they do damage differently), thus
we use different design patterns and implementations to complete such features. 

The Decorator and Strategy patterns are possible candidates to model the
effects of potions, in your opinion, which pattern would work better? 

To adhere the decorator design pattern, potions would be concrete classes of
Decorator. While any player character or mobile who can use potions owns a
concrete class of Component corresponding to the “base” potion effect, which
defaults to doing nothing. At the creation of a character it is initialized
with a concrete class of Component which returns 0. When using a potion with
non-permanent effect, the Decorator is attached to the corresponding component.
When using a potion with permanent effect, the value is directly added to the
player’s status. When calculating a player’s final status values, the chain of
Decorators is called, which return value is added to the player’s own status
value. In this way, one can detach any of the non-permanent potions at ease.
Any special ability or non-standard event during combat is not handled by the
Decorators and is directly calculated in the character class. 

To adhere the strategy pattern, one has to create a concrete Strategy for each
player status. The character owns one pointer to a virtual Strategy object for
each of their stats. When encountering a potion, the effect of the potion is
passed to the corresponding object. All the strategy object have to do is to
record the effect, and hold a collection of all the potions that it has
encountered during the game. When calculating the character’s final stats, the
original value is added to the value returned by the Strategy object, which
gives the combined effect of potions. In addition it can provide a reset
function which erases the effect of non-permanent potions.  

Both design pattern have their advantage and disadvantages. The decorator
pattern allows easier management of each potions, which provides extensibility
when we are required to access each individual potion. The strategy design is
easier to implement in the default case and uses less resource. Although it can
only handle a small amount of combinations of potions, which is not a problem
in this case.  

In my opinion, the decorator pattern would work better, as it allows us to
easily add functionalities. 

How could you generate items so that the generation of Treasure and Potions
reuses as much code as possible? That is, how would you structure your system
so that the generation of a potion and then generation of treasure does not
duplicate code? 

To reuse as much code as possible, Potions and Treasures should inherit the
same base class. Note that treasure and potion share many similarities. They
can only be picked up when players are near them, and they are immobile items
generated at the start of each floor. To do so we need them to both inherit a
virtual base class of Items (in our design UML, it is called NonLiving). The
constructor controls where the object is placed, as well as the specific type
of the potion/treasure, which is determined by an enum class type predefined.
Then we encapsulate the generation of these items in a method and let it handle
the details of generating each type of items. If we want to extent this
functionality, we have to provide another method with different parameters that
controls things such as number and proportions. 



Extra Credit Features 

In our project, we uses smart pointers to manage memory without leak. No delete
statement exists in our program.  




Final Questions 

What lessons did this project teach you about developing software in teams? 

Communication is very important to facilitate a smooth working flow. A good
design does not only increases coupling, it also makes the classes easier to
implement. If the interface of each classes are determined beforehand, each
team member only has to deal with the classes they are responsible for, not
worrying about how their classes are used by other classes. Also, the classes
that has more clients should be implemented first, since it is likely that the
design will be changed during development; except for the main loop of the
game, which should be written first to give a better idea of how the game
should be run in general.  

What would you have done differently if you had the chance to start over? 

If I could start over, I would change the design of board to be a nested class
of game controller, this allows multiple boards to be created during the game,
so player may go back to the previous floor. I would also eliminate the use of
friend classes so that all operations are made from public methods to reduce
cohesion. The display classes should be made such that it has an inclusive list
of all possible game events, so that the newAction method is not needed, and
all call made to display should only contain flags. This increases the
complexity of the class, but it also allows several game messages to be
combined into one sentence, making it more user friendly.  

Conclusion 







\end{document}
